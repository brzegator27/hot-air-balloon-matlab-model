\documentclass[a4paper,11pt]{article}
\usepackage[T1]{fontenc}
\usepackage{polski}
\usepackage[utf8x]{inputenc}
\usepackage{microtype}

\usepackage[margin=1in]{geometry}
\usepackage[]{mathtools}
\usepackage[]{amssymb}
\usepackage[]{physics}

\title{Model lotu balonu - wzory\\ analiza numeryczna}
\author{Jakub Brzegowski}
\date{Grudzień 2015}

\setlength{\parindent}{0em}
\setlength{\parskip}{2em}

\begin{document}
\maketitle

\section{Użyte wzory:}

Obliczanie przyrostu temperatury:
\[
    \dv{T}{t} = \frac{P}{C_w \cdot m}
\]
$T$ - temateratura ośrodka, \\
$t$ - czas, \\
$P$ - moc, \\
$C_w$ - ciepło właściwe, \\
$m$ - masa

\noindent \rule{16cm}{2pt}

Zależność ciśnienia od wysokości:
\[
    p = p_0 \left( 1 - \frac{L\, h}{T_0} \right)
\]

\noindent \rule{16cm}{2pt}

Siła oporu powietrza:
\[
    F_D = \tfrac{1}{2}\, \rho\, v^2\, C_D\, A
\]
$F_D$ - siła oporu aerodynamicznego \\
$\rho$ - gęstość ośrodka \\
$v^2$ - prędkość obiektu relatywna do ośrodka \\
$C_D$ - współczynnik oporu aerodynamicznego \\
$A$ - pole przekroju \\

\noindent \rule{16cm}{2pt}

\[
    n = \frac{m}{M}
\]
$n$ - liczba moli, \\
$m$ - masa substancji, \\
$M$ - masa cząsteczkowa(masa molowa) danej substancji.

\noindent \rule{16cm}{2pt}

Z równania Clapeyrona:
\[
    p\, v = n\, R\, T \\
\]
\[
    p\, \frac{m}{g} = \frac{m}{M}\, R\, T \\
\]
\[
    \rho = \frac{p\, M}{R\, T}
\]
$m$ - masa substancji, \\
$p$ - ciśnienie, \\
$v$ - objętość, \\
$\rho$ - gęstość, \\
$n$ - liczba moli gazu, będąca miarą liczby jego cząsteczek; $n = v/V$, \\
$T$ – temperatura bezwzględna, $T\, [K]  = t\, [^{\circ}C] + 273,15$ \\
$R$ – uniwersalna stała gazowa: $R = NAkB$, gdzie: $NA$ – stała Avogadra (liczba Avogadra), $kB$ – stała Boltzmanna, $R = 8,314\; \frac{J}{(mol \cdot K)}$

\noindent \rule{16cm}{2pt}

\end{document}
