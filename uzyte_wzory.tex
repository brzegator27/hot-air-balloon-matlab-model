\documentclass[a4paper,11pt]{article}
\usepackage[T1]{fontenc}
\usepackage{polski}
\usepackage[utf8x]{inputenc}
\usepackage{microtype}

\usepackage[margin=1in]{geometry}
\usepackage[]{mathtools}
\usepackage[]{amssymb}
\usepackage[]{physics}

\usepackage{array} % I use it for tabular tag.
%%%%%%%%%%%%%%%%%%%%%%%%%%%%%%%%%%%%%%%
% We can achieve alignment of dashes in equations variables descriptions by this code:
%
% \begin{tabular}{>{$}r<{$}@{\ --\ }l}
% <Variable> & <Variable description>, \\
% \end{tabular}
%
% Source: http://tex.stackexchange.com/questions/214925/center-hyphen-between-words
%%%%%%%%%%%%%%%%%%%%%%%%%%%%%%%%%%%%%%%

\title{Model lotu balonu - wzory\\ analiza numeryczna}
\author{Jakub Brzegowski}
\date{Grudzień 2015}

\setlength{\parindent}{0em}
\setlength{\parskip}{2em}

\begin{document}
\maketitle

\section{Użyte wzory:}

Siła ciężkości:
\[
    F_g = m \cdot g
\]
\begin{tabular}{>{$}r<{$}@{\ --\ }l}
    F_g     & siła ciężkości, \\
    m       & masa ciała, \\
    g       & przyspieszenie grawitacyjne. \\
\end{tabular}

\noindent \rule{16cm}{2pt}



Przyspieszenie(w ruchu prostoliniowym):
\[
    a = \dv{v}{t}
\]
\begin{tabular}{>{$}r<{$}@{\ --\ }l}
    a       & przyspieszenie, \\
    v       & szybkość ciała, \\
    t       & czas. \\
\end{tabular}

\noindent \rule{16cm}{2pt}



Prędkość chwilowa:
\[
    v = \dv{s}{t}
\]
\begin{tabular}{>{$}r<{$}@{\ --\ }l}
    v       & prędkość, \\
    s       & droga, \\
    t       & czas, \\
\end{tabular} \\
skąd:
\[
    s=\int\limits_{t_0}^{t_1} \operatorname ds =\int\limits_{t_0}^{t_1} v(t) \operatorname{dt} \text{.}
\]

\noindent \rule{16cm}{2pt} 



Obliczanie przyrostu temperatury:
\[
    \dv{T}{t} = \frac{P}{C_w \cdot m}
\]
\begin{tabular}{>{$}r<{$}@{\ --\ }l}
    T       & temateratura ośrodka, \\
    t       & czas, \\
    P       & moc, \\
    C_w     & ciepło właściwe, \\
    m       & masa.
\end{tabular}

\noindent \rule{16cm}{2pt}



Zależność pomiędzy liczbą moli, a masą i masą cząsteczkową substancji:
\[
    n = \frac{m}{M}
\]
\begin{tabular}{>{$}r<{$}@{\ --\ }l}
    n       & liczba moli, \\
    m       & masa substancji, \\
    M       & masa cząsteczkowa(masa molowa) danej substancji.
\end{tabular}

\noindent \rule{16cm}{2pt}



Gęstość powietrza w kopercie balonu: \\
z równania Clapeyrona mamy:
\[
    p\, v = n\, R\, T \\
\]
\[
    p\, \frac{m}{g} = \frac{m}{M}\, R\, T \\
\]
\[
    \rho = \frac{p\, M}{R\, T}
\]
\begin{tabular}{>{$}r<{$}@{\ --\ }l}
    m       & masa substancji, \\
    p       & ciśnienie, \\
    v       & objętość, \\
    \rho    & gęstość, \\
    n       & liczba moli gazu, będąca miarą liczby jego cząsteczek; $n = v/V$, \\
    T       & temperatura bezwzględna, $T\, [K]  = t\, [^{\circ}C] + 273,15$ \\
    R       & uniwersalna stała gazowa: $R = NAkB$, gdzie: $NA$ – stała Avogadra (liczba Avogadra), $kB$ – stała Boltzmanna, $R = 8,314\; \frac{J}{(mol \cdot K)}$. \\
\end{tabular}

\noindent \rule{16cm}{2pt}



Siła wyporu:
\[
    F_w = \rho \cdot g \cdot V
\]
\begin{tabular}{>{$}r<{$}@{\ --\ }l}
    F_w     & siła wyporu, \\
    \rho    & gęstość cieczy lub gazu, w którym znajduje się ciało, \\
    g       & przyspieszenie grawitacyjne, \\
    V       & objętość wypieranego płynu równa objętości części ciała zanurzonego w ośrodku. \\
\end{tabular}

\noindent \rule{16cm}{2pt}



Zależność ciśnienia od wysokości:
\[
    p = p_0 \cdot \left( 1 - \frac{L \cdot h}{T_0} \right)^{\frac{g \cdot M}{R \cdot L}}
\]
\begin{tabular}{>{$}r<{$}@{\ --\ }l}
    p_0     & standardowe ciśnienie atmosferyczne na poziomie morza, \\
    L       & gradient adiabatyczny temperatury, \\
    h       & wysokość nad poziomem morza, \\
    T_0     & standardowa temateratura na poziomie morza, \\
    g       & przyspieszenie ziemskie, \\
    M       & masa molowa suchego powietrza, \\
    R       & uniwersalna stała gazowa. \\
\end{tabular}

\noindent \rule{16cm}{2pt}



Siła oporu powietrza:
\[
    F_D = \tfrac{1}{2}\, \rho\, v^2\, C_D\, A
\]
\begin{tabular}{>{$}r<{$}@{\ --\ }l}
    F_D     & siła oporu aerodynamicznego, \\
    \rho    & gęstość ośrodka, \\
    v^2     & prędkość obiektu relatywna do ośrodka, \\
    C_D     & współczynnik oporu aerodynamicznego, \\
    A       & pole przekroju.
\end{tabular}

\noindent \rule{16cm}{2pt}



\end{document}
